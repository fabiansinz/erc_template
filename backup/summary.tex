Sensory systems are our window to the world around us and provide the basis for our decisions and behavior. 
However, sensory processing is not a one-way street:
Motor behavior and the internal state of an animal change sensory processing, not only through changes in gain but also the selectivity of neurons. 
% Previous work has shown that these changes can temporarily sharpen the focus of the brain to certain aspects of the environment.
% These changes are temporary which suggests that sensory certainty is a limited resource that has to be spent wisely depending on the behavioral context.
However, because the vast majority of studies are performed in head-fixed animals with limited behavioral options, it is an open question how and when the brain adjusts sensory systems during real-world, unconfined behavior. 
I hypothesize that changes in neuronal selectivity happen frequently during free behavior and adapt sensory systems to the current behavioral needs by decreasing uncertainty about relevant aspects of the world.

I propose to investigate this idea by building data-driven functional twin models of mice in a digitized version of a real environment using state of the art machine learning and a massive dataset of recordings from the entire visual cortex of head-fixed and freely behaving mice under spontaneous and goal-driven behavior.
I will use this twin to characterize which neurons change their tuning, how and under which behavioral context they change it, and how behavior would be expected to change if these mechanisms were turned off.
If successful, this project can change our view on how the visual system adapts with behavior in real world tasks. 
The proposed computational framework is straightforward to generalize to other areas, stimulus modalities, and species, and will be an enabling tool for neuroscience in unconfined animals in natural environments.
%, taking an important step towards a standard model of systems neuroscience.
