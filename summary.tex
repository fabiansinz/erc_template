Sensory systems are our window to the world and form the basis for our decisions and behavior. 
However, sensory processing is not a one-way street:
Motor behavior and the internal state of an animal can alter how sensory signals are processed.
Even in head-restrained animals, previous studies found a rich influence of motor behavior on neural activity. 
How and when the brain adjusts sensory processing during unconfined real-world  behavior in complex natural environments is currently unknown. 
I hypothesize that, under these conditions, the brain dynamically adjusts sensory processing to the current behavioral context not only by changing the gain, but the selectivity of sensory neurons. 
I further hypothesize that these changes momentarily increase the reliability of relevant environmental aspects in neuronal representations. 

I will investigate this idea in the visual system by building deep data-driven functional twin models of visual cortex and behavior of mice in digital replicas of real experimental environments.
The models will be trained on a massive dataset of recordings from excitatory neurons from nine areas of visual cortex in head-fixed and freely behaving mice under spontaneous and goal-driven behavior.
I will use these twins to discover which neurons change their tuning, how and under which behavioral context they change it, and how the animal's behavior would be different if these mechanisms were shut down.
If successful, this project will not only change our view on how the visual system adapts to behavior under natural conditions, but also yield a computational framework with a vast range of applications in neuroscience. 
It is straightforward to generalize to other areas, stimulus modalities, and species, and will be an enabling tool to study how the brain makes sense of the environment in unconfined behaving animals.

% in real world tasks
 % -- not only changing the gain, but also the selectivity of single sensory neurons